% !Mode:: "TeX:UTF-8"

\chapter{引言}
软件工具包用来进行复杂的空间动态系统的非线性分析越来越多地使用基于Web的网络平台,以实现他们的用户界面,科学分析,分布仿真结果和科学家之间的信息交流。对于许多应用系统基于Web访问的非线性分析模拟软件成为一个重要组成部分。网络硬件和软件方面的密集技术变革提供了比过去更多的自由选择机会。因此,WEB平台的合理选择和发展对整个地区的非线性分析及其众多的应用程序具有越来越重要的意义。现阶段的WEB发展的特点是出现了大量的开源框架。框架将Web开发提到一个更高的水平,使基本功能的重复使用成为可能和从而提高了开发的生产力。

在某些情况下,开源框架没有提供常见问题的一个解决方案。出于这个原因,开发在开源框架的基础上建立自己的项目发展框架。本文旨在描述是一个基于Java的框架,该框架利用了开源框架并有助于开发基于Web的应用。通过分析现有的开源框架,本文提出了新的架构,基本环境及他们用来提高和利用其他一些框架的相关技术。架构定义了自己开发方法,其目的是协助客户开发和事例项目。

应用程序设计应该关注在项目中的重复利用。即使有独特的功能要求,也有常见的可用模式使用,这使得它们的设计和开发能重用。本文介绍了一个“自定义”框架,这个框架用来定义能被开发者使用的相同的应用问题和定义设计模式。这个框架,我们将称之为某某开发框架,提供了一套模式和工具,建立了行业最佳实践,使之适合常见的应用问题。它提供了一个从表示到集成的应用程序开发堆栈。本文阐明了这些应用问题和模式,工具和行业最佳实践。某某开发框架可以根据各种项目的需求进行定制。它的开发和配置是基于诸如Struts、Spring、Hibernate和JUnit之类的各种框架和工具。
