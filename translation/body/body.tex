% !Mode:: "TeX:UTF-8"

\title{浙江工业大学本科生毕业设计外文翻译模板}
\date{}
\maketitle

\noindent{\hei 摘要:}摘要是论文内容的高度概括,应具有独立性和自含性,即不阅读论文的全文,就能获得必要的信息。
摘要应包括本论文的目的、主要研究内容、研究方法、创造性成果及其理论与实际意义。
摘要中不宜使用公式、化学结构式、图表和非公知公用的符号和术语,不标注引用文献编号。避免将摘要写成目录式的内容介绍。\\
{\hei 关键词:}\quad 关键词~1;关键词~2;关键词~3;……;关键词~6(关键词总共~3~—~6~个,最后一个关键词后面没有标点符号)
%关键词~1;关键词~2;关键词~3;……;关键词~6(关键词总共~3~—~6~个,最后一个关键词后面没有标点符号)

\section{引言}
软件工具包用来进行复杂的空间动态系统的非线性分析越来越多地使用基于Web的网络平台,以实现他们的用户界面,科学分析,分布仿真结果和科学家之间的信息交流。对于许多应用系统基于Web访问的非线性分析模拟软件成为一个重要组成部分。网络硬件和软件方面的密集技术变革提供了比过去更多的自由选择机会。因此,WEB平台的合理选择和发展对整个地区的非线性分析及其众多的应用程序具有越来越重要的意义。现阶段的WEB发展的特点是出现了大量的开源框架。框架将Web开发提到一个更高的水平,使基本功能的重复使用成为可能和从而提高了开发的生产力。
在某些情况下,开源框架没有提供常见问题的一个解决方案。出于这个原因,开发在开源框架的基础上建立自己的项目发展框架。本文旨在描述是一个基于Java的框架,该框架利用了开源框架并有助于开发基于Web的应用。通过分析现有的开源框架,本文提出了新的架构,基本环境及他们用来提高和利用其他一些框架的相关技术。架构定义了自己开发方法,其目的是协助客户开发和事例项目。
应用程序设计应该关注在项目中的重复利用。即使有独特的功能要求,也有常见的可用模式使用,这使得它们的设计和开发能重用。本文介绍了一个“自定义”框架,这个框架用来定义能被开发者使用的相同的应用问题和定义设计模式。这个框架,我们将称之为某某开发框架,提供了一套模式和工具,建立了行业最佳实践,使之适合常见的应用问题。它提供了一个从表示到集成的应用程序开发堆栈。本文阐明了这些应用问题和模式,工具和行业最佳实践。某某开发框架可以根据各种项目的需求进行定制。它的开发和配置是基于诸如Struts、Spring、Hibernate和JUnit之类的各种框架和工具。
\section{开发框架的主要技术}
\subsection{代码和配置的层与层之间的分离}
Web应用程序有各种设计问题,如表现,商业逻辑,数据存取和安全性。不同的代码层的分离设计有如下几个方面的优势,如:便于维修,实施良好设计模式的能力,选择专门的工具的能力和具体问题的解决技术。将一个项目进行层与层之间的分离导致了这些层之间的依赖关系。例如,一个简单的使用案例,它涉及数据的输入和查询通常必须整合表示,业务逻辑和数据访问以达到所需的功能[3] 。因此,必须有一个明确的策略来管理这些依赖关系。开发框架包括设计模式,可复用的代码和配置文件,使开发框架尽可能地容易的被使用。这一框架使用Spring的反向控制来管理相依。 Spring框架[4]提供了一种方法整合各层成为一个应用项目。它通过Spring应用上下文来完成这一目标,这是一个对象之间管理依赖策略。Spring使用的依赖注入和拦截技术介绍如下。
我们所写的代码依赖于使用的对象。它负责创建这些对象。这可能导致紧耦合的,但我们希望我们的代码是松散耦合。依赖注入是一个技术,可以帮助我们实现这一目标。依赖注入是反向控制(IOC)的一种形式[5]。当应用程序使用依赖注入时,代码将变得更加清洁和容易。这就是松耦合,从而更容易配置和测试。开发框架使用了多个Spring应用背景文件来定义层与层之间的依赖关系。方法拦截是面向方面编程(AOP)概念[6]。Spring AOP方法拦截是通过JDK动态代理来实现的。开发框架使用Spring AOP来管理问如交易管理和性能监测等问题。
开发框架包括两个不同的部分:代码和配置。代码位于一个特定的应用层,并侧重于某一特定条件中的应用解决方案。这可能要与数据库交互,或将数据显示给屏幕。配置将应用的各个层联系在一起。从代码中分离出配置使我们能够独立管理配置,使我们在同一代码基础上方便的进行不同的配置。例如,数据访问对象(DAO)知道它是使用JDBC通过数据源来连接一个数据库的,但它不知道该数据源是如何实现的。它可能是一个Java命名和目录接口(JNDI上下文或是来自驱动程序。它可以指向远程数据库或本地数据库。无论数据来自何处,DAO执行操作数据源的方式是相同的。同样,服务对象可能依赖于DAO ,但不知道DAO是如何实现,可能通过Hibernate,直接的JDBC ,或Web服务。互动服务对象与DAO有相同的方式,而不管DAO的实现。
