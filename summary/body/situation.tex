% !Mode:: "TeX:UTF-8"

\chapter{国内外研究现状及难点}
在国外,教学质量管理已有90年的历史。以美国为代表的许多国家,如澳大利亚、英国、加拿大、比利时等国都相继采用学生评教来评价教师的教学效果。以美国为例,20世纪70年代初,美国教育委员会的一个调查结果表明,在被调查的669所高等学校中,大约有65%的高校在系一级机构中允许学生对教学进行评价,到80年代以后,学生评教不但成为大学教学评价的二个重要组成部分,且评价技术越来越现代化。目前,许多大学已经开发使用了基于网络的学生评教系统,如华盛顿大学的IAS(Instructional Assessment System)、亚利桑那大学的TCE(Teacher-Course Evaluation)、堪萨斯州立大学的IDEA(Individual Development and Educational Assessment)等[2],这些系统通过校园网络实施教学评价,取得了较好的效果。美国等国家已经有网上申报、网上专家评审的系统,基于网络的申报管理信息系统国外已进入实用研究阶段,大量的投入到各种项目的网上申报、网上评审的实际运用中,提高的项目申报申批的效率,取得了重大的经济效益。

在我国,学生评教的发展经历了定性评教为主和定量评教为主等阶段,比较规范的科学的学生评教活动应当说是伴随科学的高教评估活动的兴起而逐步形成并得到良好发展的。1985年之后开展的各种高教评估试点活动,都离不开对教学质量特别是课堂教学质量的评估,对于后者除了用统测的办法之外,另一个更为可行的办法就是学生评教[4]。我国的学生评教活动始于20世纪80年代初,特别是从1987年起,随着教师职称评定工作日益规范化,许多高校对教师的教学提出了越来越高的要求,学生评教活动开展得越来越普遍。2001年教育部4号文件——《关于加强高等学校本科教学工作提高教学质量的若干意见》出台后,学生评教在全国普通高校更是得到了广泛的开展,评教方式和技术手段也逐步得到了改进。各种基于网络的学生评教信息系统也取得了较大的进展。但相比于国外而言,我国的教学质量网上管理系统的开发还有一定的距离,而且在国家与省级之间也存在着一定的差距。国家教学质量与教学改革工程项目的立项都已经实现网上申报、网上评审,种类科技项目一般也都已经实行网上申报、网上评审。但浙江省高教处的项目管理工作基本上都是基于传统的纸质材料,已经严重落后于电子政务建设的步伐,管理很难全面地了解把握各类建设项目的立项、建设进展等情况。这样既不符合申报材料电子化的趋势,也限制了项目评审专家的选择、项目评审的公平、公正。因此,在国外已进入实用研究阶段时,国内还处于设想开发的初级阶段。

目前,该领域研究的难点主要有:基于互联网申报、评审的管理模式的研究,对于多层次、多级别的管理层,针对复杂多样的网络环境,提出一种适合于互联网的申报、评审管理模式;数据的安全性,对于数据的远程传输、备份及权限的设计、加密算法等;各种网上结构化、非结构化表格的处理与管理,面对不同级别、不同类型的项目立项报告书,格式转换、存储、传输、输出和归档管理,以及查询、修改、分类统计和输出;不同层次的机构组织的通讯、协调管理,有关项目需要主管部门先评审或者主管部门先排序,再上报省教育厅正式评审,而有关项目不需要主管部门先评审可直接报省科技厅评审,这样系统必然对不同项目进行不同级别的管理。
