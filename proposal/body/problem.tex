% !Mode:: "TeX:UTF-8"

\chapter{研究开发的基本内容、目标,拟解决的主要问题或技术关键}
\section{研究目标}
在对比国内外教学质量工程申报评审系统的基本上,在研究国外内类似系统的设计实现上,提出自己的设计与实现。在当前教育优先发展的情况下,国家实施科教兴国战略,在这种情况下,教学质量当然也是非常重要的一个因素,关于如何提高教学质量,国家和学校都做了一些探索。特别在当信息技术如此普及的时代,借助信息技术来提高教学质量已是一种普遍的做法,国外已经在这方面走在了前头。本课题的研究目标定位于利用J2EE技术来实现教学质量工程申报评审系统的实现,特别是应用J2EE中的一些关键技术和框架,如Hibernate、Spring、Spring MVC。

\section{研究的基本内容}
由于整个系统的结构庞大,开发工作量大,所以本研究的基本内容并不定位于整个系统的设计与实现上。相反,本研究的基本内容是教学质量工程申报评审系统中的申报子系统功能模块上。
申报子系统的主要功能是根据教育厅发布的项目申报指南和限额,项目申报单位(学校)组织本校教师集中进行项目的申报及对项目的初审。
本研究的具体容包括:

(1)	信息发布\par
信息发布的主要功能是申报通知、申报指南等信息发布,主要是文字内容和相关文档附件。发布信息只能由教育厅主管部门人员进行。对于撰写完毕的信息,可以存入草稿箱中,等待用户修改后发布。对信息提供添加、删除、修改功能。

(2)	项目申报\par
项目申报的主要功能是项目申报人根据学校分发的项目申报密钥进行项目申报书的填写。申报项目的类型和项目的名称已由学校事先录入,申报人不得更改。申报人需要填写在线项目申报简表,上传项目申报书(PDF格式)。填写中可对内容保存、提供修改功能。最后,把申报简表和申报书一起提交到学校。

项目的申报是项目申报子系统中的一个重要的功能,也是该子系统的核心功能。主要包括两大模块填写申报书和提交申报书。

申报人必须按规定在线填写申报简表,按申报书要求离线填写项目申报书,然后把申报简表和项目申报书提交学校。项目的名称和类别已经由学校指定,申报人不得修改;如需修改,必须由学校进行修改。
项目申报信息的填写可以中途保存或填写完毕后提交。提交的申报项目将不能修改。在项目申报的有效时间段内,用户都可以凭密钥登陆系统。项目申报书提交后就不能修改,但可查阅。系统不设置自动提交项目申报材料功能,提交工作由申报人手工操作,并进行确认。

申报简表和申报书核对无误后,申报人把申报简表和项目申报书提交到学校。项目申报书提交后就不能修改,若要修改,需要由学校先进行退回操作。

(3)	学校申报管理\par
各申报单位(学校)负责管理本单位的项目申报工作,并对项目进行初审。学校根据教育厅下达的种类项目的申报限额和申报截止时间,建立本校的具体申报项目和相应的用户密钥,完成本校申报项目的初审并报送教育厅。若申报时间逾期,学校将不能向教育厅提交项目申报,除非教育厅给予再次授权开通。学校在向教育厅正式提交项目申报前,可对申报人所提交的申报材料进行查阅、审核,可以把申报材料退回申报人进行修改,但当正式提交教育厅后,就不能再对申报材料进行修改操作。

在申报系统中,每个学校只能查看本校申报的项目。

新建申报项目。学校在教育厅所授权指定的项目类型中进行项目申报的新建。新建的申报项目数量不得超过教育厅设定的本校申报限额。新建申报项目需要指定项目的名称、项目所属的学科门类(便于项目分组和专家匹配)和指定申报用户密钥。对新建项目申报以列表形式显示,并标示为待填报状态。列表显示新建申报项目的公共属性(如项目类型、项目所属学科门类、项目申报人、用户名)。教师申报用户凭学校分发的密钥进行项目申报书的填写。申报系统对列表中的新建申报项目,提供查看、修改、删除操作。

待初审项目。在申报用户提交后,项目即转入等初审状态。学校对申报项目进行初审,对需要修改的申报项目,可以退回申报人进行修改。

初审通过项目。对初审通过的项目,标示状态为初审通过。对初审通过的项目,提供同类别申报项目的整批提交操作。系统不提供逐个项目单独提交的功能。系统提供对初审项目的优先排序等特殊标记功能。
已提交项目。向教育厅提交的本校申报项目。对已经提交的项目标示已提交教育厅的状态。对项目的提交,只允许提交一次,且前提条件为提交项目的总数不得超过限额,对于小于限额数量的提交操作,给出提醒信息。若一次提交了部分项目,之后又要再提交一些项目,则需要管理员将已经提交的项目退回,申报单位一次性提交全部项目。

(4)	申报设定\par
申报设定的主要功能是对项目申报类别和每个类别各学校相应申报数量的管理。只有启用的申报类别,在学校申报管理中出现。同时可以启用多个申报类别。设定每个申报类别的申报评审时间限制。

\section{需要解决的技术难点}
Spring MVC,Hibernate,Spring框架的整合使用。\par
Ajax技术的使用。\par
密钥的生成与管理。\par
