% !Mode:: "TeX:UTF-8"

\chapter{选题的背景与意义}
\section{研究开发的目的}
随着教育部教学质量与教学改革工程建设工作的展开,浙江省教育厅也设立了一批相应的教学质量与教学改革项目。在省级项目立项、国家级项目推荐、已经立项的各类项目的管理与检查等方面,目前浙江省教育厅高等教育处没有相应的电子化的项目管理、项目申报评审系统。为提高管理水平和效率,迫切需要建设项目管理的硬件平台和设计开发一套符合我省教学质量工程项目实际需要的软件系统。建设“浙江省教育厅教学质量与教学改革项目申报管理系统”(暂名,以下简称项目申报管理系统)将能有效地促进高等教育处的管理工作。

(1) 提高管理的质量与效率。\par
目前的高教处的项目管理工作基本上都是基于传统的纸质材料,已经严重落后于电子政务建设的步伐,管理者很难全面地了解把握各类建设项目的立项、建设进展等情况。申报管理系统建设后,将能提供快速、准确、全面的种类项目的立项与建设情况。

(2) 实现项目的网上申报、网上评审。\par
国家教学质量与教学改革工程项目的立项都已经实行网上申报、网上评审,各类科技项目一般也都已经实行网上申报、网上评审。我省教学质量工程项目的申报与评审仍然是传统的纸质材料文件方式。这样既不符合申报材料电子化的趋势,也限制了项目评审专家的选择、项目评审的公平、公正。实行项目的网上申报、网上评审,将大大提高项目评审工作的效率与评审专家的选择范围。

(3) 展示教学质量与教学改革成果,发挥示范作用。\par
在现在基于纸质材料的管理体制下,建设与改革项目的成果难以进行比较大面积的宣传与交流。在拟建设的系统中,设立成果展示平台,发挥优秀成果的示范作用。

\section{国内外研究发展现状}
在国外,教学质量管理已有90年的历史。以美国为代表的许多国家,如澳大 利亚、英国、加拿大、比利时等国都相继采用学生评教来评价教师的教学效果。以美国为例,20世纪70年代初,美国教育委员会的一个调查结果表明,在被调查的669所高等学校中,大约有65%的高校在系一级机构中允许学生对教学进行评价,到80年代以后,学生评教不但成为大学教学评价的二个重要组成部分,且评价技术越来越现代化。目前,许多大学已经开发使用了基于网络的学生评教系统,如华盛顿大学的IAS(Instructional Assessment System)、亚利桑那大学的TCE(Teacher-Course Evaluation)、堪萨斯州立大学的IDEA(Individual Development and Educational Assessment)等[2],这些系统通过校园网络实施教学评价,取得了较好的效果。美国等国家已经有网上申报、网上专家评审的系统,基于网络的申报管理信息系统国外已进入实用研究阶段,大量的投入到各种项目的网上申报、网上评审的实际运用中,提高的项目申报申批的效率,取得了重大的经济效益。

在我国,学生评教的发展经历了定性评教为主和定量评教为主等阶段,比较规范的科学的学生评教活动应当说是伴随科学的高教评估活动的兴起而逐步形成并得到良好发展的。1985年之后开展的各种高教评估试点活动,都离不开对教学质量特别是课堂教学质量的评估,对于后者除了用统测的办法之外,另一个更为可行的办法就是学生评教。我国的学生评教活动始于20世纪80年代初,特别是从1987年起,随着教师职称评定工作日益规范化,许多高校对教师的教学提出了越来越高的要求,学生评教活动开展得越来越普遍。2001年教育部4号文件——《关于加强高等学校本科教学工作提高教学质量的若干意见》出台后,学生评教在全国普通高校更是得到了广泛的开展,评教方式和技术手段也逐步得到了改进。各种基于网络的学生评教信息系统也取得了较大的进展。但相比于国外而言,我国的教学质量网上管理系统的开发还有一定的距离,而且在国家与省级之间也存在着一定的差距。国家教学质量与教学改革工程项目的立项都已经实现网上申报、网上评审,种类科技项目一般也都已经实行网上申报、网上评审。但浙江省高教处的项目管理工作基本上都是基于传统的纸质材料,已经严重落后于电子政务建设的步伐,管理很难全面地了解把握各类建设项目的立项、建设进展等情况。这样既不符合申报材料电子化的趋势,也限制了项目评审专家的选择、项目评审的公平、公正。因此,在国外已进入实用研究阶段时,国内还处于设想开发的初级阶段。
